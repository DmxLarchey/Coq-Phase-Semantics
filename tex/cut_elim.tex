\documentclass[xcolor=pdftex,graphicx=pdftex,12pt]{beamer}

\usepackage[utf8]{inputenc}

\usepackage{deduc}
\usepackage{export}
\usepackage{tree}
\usepackage{domdeduc}

\makeatletter

\newbox\@tempboxb

\def\centertwo#1#2{%
  \bgroup
  \setbox\@tempboxa\hbox{#1}%
  \setbox\@tempboxb\hbox{#2}%
  \@tempdima=\linewidth
  \@tempdimb=\wd\@tempboxa
  \advance\@tempdima-\@tempdimb
  \@tempdimb=\wd\@tempboxb
  \advance\@tempdima-\@tempdimb
  \divide\@tempdima 3%
  \hbox{\box\@tempboxa\hskip\@tempdima\box\@tempboxb}%
  \egroup}

\makeatother

%%%%%%%%%%%%%%%%%%%%%%%%%%%%%%%%%%%%%%%%%%%%%%
% Macros for proof trees
%%%%%%%%%%%%%%%%%%%%%%%%%%%%%%%%%%%%%%%%%%%%%%

\newcommand{\vertcenter}[1]{\hbox{$\vcenter{\hbox{#1}}$}}

\newcommand{\pftreesmacs}{%
  \def\VC##1{\hbox{$\vcenter{\hbox{##1\unskip\buildproof{}}}$}}%
  \def\CR##1##2{\createrule{##1}{$##2$}}%
  \def\CA##1##2{\createaxiom{##1}{$##2$}}%
}

%\usepackage[pdftex]{graphicx}
\DeclareGraphicsRule{*}{mps}{*}{}

\newcommand\ofr[2]{\only<#1>{\includegraphics{th.#2}}}

\usepackage{tabularx}

\usepackage{amssymb,amsmath}
\usepackage{latexsym,stmaryrd}
\usepackage{mathtools}

\usepackage{inferance}
\renewcommand\frule[1]{\ensuremath{\langle#1\rangle}}

\usepackage{color}
\usepackage{url}

\makeatletter
\def\url@domistyle{%
  \@ifundefined{selectfont}{\def\UrlFont{\sffamily}}{\def\UrlFont{\sffamily}}}
\makeatother

\newcommand{\TaH}{\text{\sf T\&H}}

\newcommand{\fileurl}{https://github.com/DmxLarchey/Coq-Phase-Semantics}
\newcommand{\filename}[1]{{\href{\fileurl/#1}{\sffamily #1}}}
%\newcommand{\filename}[1]{{\urlstyle{domi}\nolinkurl{#1}}}

\newcommand{\coq}[1]{\ensuremath{\mathalpha{\mathtt{#1}}}}

\newcommand{\mathbinop}[3]{\mathop{\mathop{#1}{#2}}{#3}}

\newcommand{\cst}[1]{\coq{cst}_{#1}}
\newcommand{\zero}{\coq{zero}}
\newcommand{\suc}{\coq{succ}}
\newcommand{\proj}[1]{\coq{proj}_{#1}}
\newcommand{\comp}{\mathbinop{\coq{comp}}} 
\newcommand{\rec}{\mathbinop{\coq{rec}}}
\newcommand{\umin}{\mathop{\coq{min}}}
\newcommand{\ra}[1]{\mathcal{A}_{#1}}

\newcommand{\ot}{\mathrel{:}}
\newcommand{\nat}{\coq{nat}}
\newcommand{\pos}{\mathop{\coq{pos}}}

\newcommand{\fst}{\coq{fst}}
\newcommand{\nxt}[1]{\mathalpha{\coq{nxt}\,{#1}}}
%\newcommand{\vect}{\mathbinop{\coq{vec}}}
\newcommand{\vect}[2]{#1^{#2}}
\newcommand{\pton}[1]{\mathalpha{\overline{#1}}}

%\newcommand{\vnil}{\coq{vnil}}
\newcommand{\vcons}{{\#}}
\newcommand{\coord}[2]{#1_{#2}}
\newcommand{\sem}[1]{\llbracket{#1}\rrbracket}
\newcommand{\us}{\_}

\newcommand{\NN}{\ensuremath{\mathbb N}}
\newcommand{\pfun}{\mathbin{{-}\mskip -3mu{\rightharpoondown}}}
\renewcommand{\pfun}{\mathbin{\rightharpoondown}}
\newcommand{\fun}{\mathbin{{-}\mskip -3mu{\rightarrow}}}

\newcommand{\cfun}{\mathbin{\rightarrow}}
\newcommand{\Prop}{\coq{Prop}}
\newcommand{\Set}{\coq{Set}}
\newcommand{\Type}{\coq{Type}}
\newcommand{\cdef}{\mathrel{\mathtt{:=}}}
\newcommand{\mt}{\mathrel{\mapsto}}
%\newcommand{\ctimes}{\mathbin{\mathtt{*}}}
\newcommand{\ctimes}{\mathbin{\times}}
\newcommand{\cF}{\coq{False}}
\newcommand{\cT}{\coq{True}}
\newcommand{\dect}{\coq{decidable\_t}}
\newcommand{\Acc}{\coq{Acc}}
\newcommand{\cmid}{\mathrel{\&}}
\newcommand{\iter}[1]{\ensuremath{\mathit{f}^{#1}}}

\renewcommand{\leq}{\mathrel{\leqslant}}
%\newcommand{\lnot}{\mathop{\neg}}
\newcommand{\conj}{\mathrel{\wedge}}
\newcommand{\disj}{\mathrel{\vee}}
\renewcommand{\iff}{\mathrel\Longleftrightarrow}
\renewcommand{\implies}{\mathrel\Rightarrow}
\newcommand{\seq}{\mathrel\vdash}

\newcommand{\eqdec}{\mathrel{=^?_X}}
\newcommand{\goal}[1]{\ensuremath{\mathbb{\color{teal}G}^?_{#1}}}

\newcommand{\bargen}[1]{\mathalpha{\coq{bar}_{\coq{#1}}}}
\newcommand{\cbar}{\mathalpha{\coq{bar}}}
\newcommand{\bartl}{\cbar}
\newcommand{\barth}{\cbar}
\newcommand{\barin}{\bargen{in}}
\newcommand{\barpe}{\bargen{pe}}
\newcommand{\cS}{\coq{S}}

\newcommand{\extract}{\textsc{extr}}

\newcommand{\carrowlft}{\mbox{${-}\mskip -4mu{\lbrack}$}}
%\newcommand{\carrowrt}{\mbox{$\rbrack\mskip -9mu\rightarrow$}}
\newcommand{\carrowrt}{\mbox{${\rangle}\mskip -4mu{\rangle}$}}

\DeclarePairedDelimiter{\carrow}{\carrowlft}{\carrowrt}
\DeclarePairedDelimiter{\brackets}{\lbrack}{\rbrack}

\newcommand{\bstep}[4]{\brackets{{#2};{#3}}\mathrel{#1}#4}
\newcommand{\cwoc}[4]{\bstep\rightsquigarrow{#1}{#2}{#4}}
\newcommand{\cwc}[4]{\bstep{\carrow{#3}}{#1}{#2}{#4}}

\newcommand{\lwith}{\mathbin{\text{\upshape \&}}}
\newcommand{\lpar}{\mathbin{\bindnasrepma}}
\newcommand{\lplus}{\mathbin{\varoplus}}
\newcommand{\ltime}{\mathbin{\varotimes}}
\newcommand{\lunit}{\mathalpha{\epsilon}}
\newcommand{\limp}{\mathbin{\multimap}}

\newcommand{\ltop}{\mathalpha{\top}}
\newcommand{\lbot}{\mathalpha{\bot}}
\newcommand{\lbang}{\mathop{!}}

\newcommand{\cperm}{\mathrel{\sim_p}}
\newcommand{\cl}{\mathop{\mathrm{cl}}}
\newcommand{\mmult}{\mathbin{\bullet}}
\newcommand{\mimp}{\mathbin{\multimap\mskip -14mu\bullet}}
\newcommand{\munit}{\mathalpha{\mathsf e}}
\newcommand{\mcomp}[3]{{#1}\mmult{#2}\mathop{\triangleright}{#3}}
%\newcommand{\sem}[1]{\mathopen{\llbracket}#1\mathclose{\rrbracket}}
\newcommand{\set}[1]{\mathcal{#1}}

\newcommand{\pable}{\mathrel{\models}}


\newcommand{\leftcoqspace}{\hskip 0.0em}

\def\displayedcoq{$$\hbox to \textwidth\bgroup\leftcoqspace$\displaystyle}
\def\enddisplayedcoq{$\hfil\egroup$$}
%\newenvironment{displayedcoq}
%  {$$\hbox to \textwidth\bgroup\leftcoqspace$\displaystyle}
%  {$\hfil\egroup$$}

\newcommand{\onelinethmold}[2]{%
  $$\begin{tabularx}{\textwidth}{>{$\displaystyle}X<{$\hfil}}%
    \coq{#1}~~#2
  \end{tabularx}$$}

\newcommand{\onelinethm}[2]
  {$$\hbox to \textwidth\bgroup$\displaystyle\begin{array}{@{\leftcoqspace}l}\coq{#1}~~#2\end{array}$\hfil\egroup$$}

\newcommand{\twolinesinductive}[2]{%
  \onelinethm{Inductive}{#1\\[0.4ex]\qquad|~#2}}

\newcommand{\displayedthm}[3]{%
  $$\begin{tabularx}{\textwidth}{@{}>{\leftcoqspace}X<{\hfil}}
    $\displaystyle\coq{#1}~~#2$\\[0.4ex]
    \hfil$\displaystyle#3$
  \end{tabularx}$$}

\title{Mechanizing Cut-Elimination in Coq\\ via Relational Phase Semantics}
\author{Dominique Larchey-Wendling}
\institute{Université de Lorraine, LORIA, CNRS, Nancy, France}
\date{Preuves de logique linéaire sur machine, ENS-Lyon, Dec. 18, 2018}

\newcommand\badcolor{\color{red}}
\newcommand\goodcolor{\color{blue}}

%\usetheme{Warsaw}
%\usetheme{Madrid}
\usetheme{Goettingen}

\begin{document}

\begin{frame}\titlepage\end{frame}

\section{Introduction to ILL}

\begin{frame}

\frametitle{Overview of the talk}

\begin{itemize}
\item Linear logic introduced by Girard
  \begin{itemize}
  \item both classical and intuitionistic
  \end{itemize}
\item ILL via its sequent calculus
  \begin{itemize}
  \item separate multiplicatives (${\ltime},{\limp},{\lunit}$)
  \item from additives (${\lwith},{\lplus},\lbot, \ltop$)
  \item multiplicatives split the context
  \item additives share the context
  \end{itemize}
\item formulas cannot be freely duplicated or discarded
  \begin{itemize}
  \item no weakening (C) or contraction rule (W)
  \item exponentials $\lbang A$ re-introduce controlled C\&W
  \item generally undecidable
  \end{itemize}
\item Mechanized cut-elimination via phase semantics
  \begin{itemize}
  \item A relational phase semantics (no monoid)
  \item via Okada's lemma, both in \Prop\ and \Type
  \end{itemize}
\end{itemize}

\end{frame}

\section{Sequent calculus}

\begin{frame}
  
\frametitle{ILL sequent calculus (multiplicatives)}

\centerline{%
\begin{tabular}{cc}
  \infer1[\ltime_L]   {A,B,\Gamma \seq C} 
                      { A \ltime B, \Gamma\seq C}
& \infer2[\ltime_R]   {\Gamma \seq A}{\Delta \seq B}
                      {\Gamma, \Delta \seq A \ltime B} \\
%
  \infer2[\limp_L]    {\Gamma \seq A}{B,\Delta \seq C}
                      {A \limp B,\Gamma, \Delta\seq C}
& \infer1[\limp_R]    {\Gamma, A \seq B}
                      {\Gamma \seq A \limp B} \\
%
  \infer1[\lunit_L]   {\Gamma \seq A}
                      {\lunit,\Gamma \seq A}
&  \infer0[\lunit_R]   {\seq \lunit}\\
%
\end{tabular}}

\end{frame}

\begin{frame}
  
\frametitle{ILL sequent calculus (additives)}

\centerline{%
\begin{tabular}{cc}
%
  \infer1[\lwith^1_L] {A,\Gamma \seq C}
                      {A \lwith B,\Gamma\seq C}
%
& \infer1[\lwith^2_L] {B,\Gamma \seq C}
                      {A \lwith B, \Gamma\seq C}\\
%
  \infer2[\lwith_R]   {\Gamma \seq A}{\Gamma \seq B}
                      {\Gamma \seq A \lwith B}
%
&  \infer2[\lplus_L]  {A,\Gamma\seq C}{B,\Gamma\seq C}
                      {A \lplus B,\Gamma\seq C}\\
%
  \infer1[\lplus^1_R] {\Gamma \seq A}
                      {\Gamma \seq A\lplus B}
%
& \infer1[\lplus^2_R] {\Gamma \seq B}
                      {\Gamma \seq A\lplus B}\\
%
  \infer0[\lbot_L]    {\Gamma, \lbot \seq A} 
%
&  \infer0[\ltop_R]    {\Gamma \seq \ltop} \\
%
%
\end{tabular}}

\end{frame}

\begin{frame}
  
\frametitle{ILL (exponentials and structural)}

\centerline{%
\begin{tabular}{cc}
%
  \infer1[\lbang_L] {A,\Gamma,  \seq B}
                     {\lbang A,\Gamma \seq B}
%
& \infer1[\lbang_R] {\lbang\Gamma \seq A}
                    {\lbang\Gamma \seq \lbang A}
\end{tabular}}

\vspace{1cm}

\centerline{%
\begin{tabular}{cc}
  \infer0[\text{id}]  {A \seq A}  
%
& \infer2[\text{cut}] {\Gamma  \seq A}{A,\Delta\seq B}
                      {\Gamma, \Delta \seq B} \\[2ex]
%
  \infer1[W]         {\Gamma,  \seq B}
                     {\lbang A,\Gamma \seq B}
%
& \infer1[C]        {\lbang A,\lbang A,\Gamma \seq B}
                    {\lbang A,\Gamma \seq B}
\end{tabular}}

\vspace{0.5cm}

$$\infer1[\Gamma\cperm\Delta]{\Gamma\seq A}{\Delta\seq A}$$

\end{frame}

\section{Phase semantics}

\begin{frame}

\frametitle{Relational Phase semantics (overview)}

\begin{itemize}
\item It is an algebraic semantics
  \begin{itemize}
  \item Comparable to Lindenbaum construction
  \item Interpret formula by ``themselves'' (completeness)
  \item {\bf does not require \frule{cut}} (cut-admissibility)
  \end{itemize}
\item Usual phase semantics based on
  \begin{itemize}
  \item commutative monoidal structure (contexts)
  \item a stable closure
  \end{itemize}
\item Relational phase semantics
  \begin{itemize}
  \item a composition relation (no axiom)
  \item closure axioms absord the monoidal structure
  \end{itemize}
\end{itemize}

\end{frame}

\begin{frame}

\frametitle{Relational Phase Semantics (details)}

\begin{itemize}
\item Closure $\cl: (M\cfun\Prop)\cfun (M\cfun\Prop)$ 
  \begin{itemize}
  \item with predicates $\set X,\set Y:M\cfun\Prop$
  $$\set X\subseteq\cl \set X\quad \set X\subseteq \set Y\cfun \cl{\set X}\subseteq \cl{\set Y}\quad\cl(\cl\set X)\subseteq \cl \set X$$
  \end{itemize}
\item Composition ${\mmult}: M\cfun M\cfun M\cfun\Prop$, $\munit: M$
  \begin{itemize}
  \item extended to predicates $M\cfun\Prop$ by
  $$ \begin{array}{rcl}
     \set X\mmult \set Y & \cdef & \bigcup\{x\mmult y\mid x\in\set X, y\in\set Y\}\\ 
     \set X\mimp \set Y & \cdef & \{ z\mid z\mmult \set X\subseteq\set Y\}
   \end{array}$$
  \item $x\in\cl(\munit\mmult x)$ \quad (neutral1)
  \item $\munit\mmult x \subseteq \cl\{x\}$ \quad (neutral2)
  \item $x\mmult y\subseteq \cl(y\mmult x)$ \quad (commutativity)
  \item $x\mmult(y\mmult z)\subseteq\cl((x\mmult y)\mmult z)$ \quad (associativity)
  \end{itemize}
\medskip
\item Stability: $(\cl\set X)\mmult\set Y\subseteq\cl(\set X\mmult\set Y)$
\end{itemize}

\end{frame}

\begin{frame}

\frametitle{Rel.\ Phase Sem.\ (exponential, soundness)}

\begin{itemize}
\item Let $\set J \cdef \{x \mid x\in\cl\{\munit\} \wedge x\in\cl(x\mmult x)\}$
\item Choose $\set K\subseteq \set J$ such that $\munit\in\cl\set K$ and $\set K\mmult\set K\subseteq\set K$ 
\item Semantics for variables: $\sem\cdot:\mathsf{Var}\cfun M\cfun\Prop$
  \begin{itemize}
  \item which is closed: $\cl\sem{V}\subseteq \sem V$
  \item extended to formulas
  \end{itemize} 
\end{itemize}

\vspace{-0.5cm}

$$
\begin{array}{c}
\sem{A\ltime B}\cdef\cl(\sem A\mmult\sem B)\quad \sem{A\limp B}\cdef \sem A\mimp\sem B\\[1ex]
\sem{A\lwith B}\cdef \sem A\cap\sem B \quad \sem{A\lplus B}\cdef \cl(\sem A\cup\sem B)\\[1ex]
\sem{\lbot}\cdef \cl\emptyset\quad \sem\ltop\cdef M\quad \sem\lunit\cdef\cl\{\munit\}\\[1ex]
\sem{\lbang A} \cdef \cl(\set K\cap\sem A)\quad \sem{\Gamma_1,\ldots,\Gamma_n}\cdef\sem{\Gamma_1\ltime\cdots\ltime\Gamma_n}
\end{array}
$$

\begin{itemize}
\item Soundness: if $\Gamma\seq A$ has a proof then $\sem\Gamma\subseteq\sem A$
\end{itemize}

\end{frame}

\section{Cut-admissibility}

\newcommand{\ulist}[1]{\lfloor#1\rfloor}

\begin{frame}

\frametitle{Relational Phase Sem.\ (cut-admissibility)}


\begin{itemize}
\item A syntactic model  $M\cdef \coq{list}~\mathsf{Form}$
\item for $\Gamma,\Delta,\Theta\in M$
$$\Theta\in \Gamma\mmult \Delta \quad\iff\quad\ulist{\Gamma,\Delta}\cperm\Theta$$
\item $\set K\cdef \{\lbang\Gamma\mid\Gamma\in M\}$ ($\emptyset\in\set K$ and $\set K\mmult\set K\subseteq\set K$)
\item \emph{contextual closure}  $\cl: (M\cfun\Prop)\cfun (M\cfun\Prop)$ 
$$\Delta\in\cl\set X\quad\iff\quad \fbox{$\forall\,\Gamma\,A,\;\set X,\Gamma\pable A\cfun \Delta,\Gamma\pable A$}$$ 
\item where ${\pable}:\coq{list}~\mathsf{Form}\cfun\mathsf{Form}\cfun\Prop$
  \begin{itemize}
  \item such as provability or \emph{cut-free provability}
  \item permutations: $\Gamma\cperm\Delta\cfun \Gamma\pable A\cfun \Delta\pable A$
  \end{itemize}
\end{itemize}

\end{frame}

\newcommand{\dc}{\mathop{\downarrow}}
\renewcommand\frule[1]{#1}


\begin{frame}

\frametitle{Rules as algebraic equations}

\begin{itemize}

\item Define $\dc A\cdef \{\Gamma\mid \Gamma\pable A\}$, then $\cl(\dc A)\subseteq \dc A$ 
\item $\dc A\mmult\dc B\subseteq \dc(A\ltime B)$ iff 
$$
\infer2[\text{for any $\Gamma,\Delta$}]   {\Gamma \pable A}{\Delta \pable B}
                      {\Gamma, \Delta \pable A \ltime B}$$
\item $\ulist{A\ltime B}\in\cl\{\ulist{A,B}\}$ iff
$$
\infer1[\text{for any $\Gamma,C$}]   {A,B,\Gamma \seq C} 
                      { A \ltime B, \Gamma\seq C}$$
\item $\set K\subseteq\set J$ iff $\pable$ closed under \frule W and \frule C.
\end{itemize}

\end{frame}

\begin{frame}

\frametitle{Okada's lemma}

\begin{itemize}
\item For ${\pable}$ defined as ${\seq_{\textrm{cf}}}$ closed under cut-free ILL
$$\fbox{$\forall A,\;\ulist A\in\sem A\subseteq\dc A$}\quad\text{and}\quad \forall\Gamma,\;\Gamma\in\sem\Gamma$$ 
\item By induction on $A$, then by induction on $\Gamma$
\item By soundness, from a (cut using) proof of $\Gamma\seq A$
  \begin{itemize}
  \item we deduce $\sem\Gamma\subseteq\sem A\subseteq\dc A$
  \item hence $\Gamma \seq_{\textrm{cf}} A$
  \item hence $\Gamma\seq A$ is cut-free provable
  \end{itemize}
\item Hence a semantic proof of cut-admissibility
\end{itemize}

\end{frame}

\begin{frame}

\frametitle{Extensions, other logics, cut-elimination}

\begin{itemize}
\item Extensions to other logics:
  \begin{itemize}
  \item of course fragments of ILL, but also CLL
  \item ILL with modality, Linear time ILL
  \item Bunched Implications (BI)
  \item Relevance logic, prop.\ Intuitionistic Logic
  \item Display calculi (context = consecutions)?
  \end{itemize}
\item Computational content
  \begin{itemize}
  \item Phase semantics, contextual closure very generic
  \item $\coq{Prop} \rightsquigarrow \coq{Type}$ gives cut-elimination algo.
  \item can be extracted (you do not want to read it...)
  \end{itemize}
\end{itemize}

\end{frame}


\end{document}
